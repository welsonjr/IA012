\documentclass[pdf]{beamer}
\mode<presentation>
{
	\usetheme{Warsaw}
}
\usepackage[utf8]{inputenc}
\usepackage[portuguese]{babel}
\usepackage[T1]{fontenc}
\usepackage{amsmath}
\usepackage{amsfonts}
\usepackage{amssymb}
\usepackage{makeidx}
\usepackage{graphicx}
\usepackage{enumerate}
\author{Welson Jr}
\begin{document}
\begin{frame}{Gabarito}
\begin{block}{Exercício a}
\begin{table}
\begin{center}
\scalebox{0.60}{
\begin{tabular}{|c|cccccccc|cccccccc|cccccccc|}
\hline 
\textbf{Texto} & \multicolumn{8}{c|}{1} & \multicolumn{8}{c|}{2} & \multicolumn{8}{c|}{3} \\ \hline
\textbf{ASCII} & \multicolumn{8}{c|}{31} & \multicolumn{8}{c|}{32} & \multicolumn{8}{c|}{33} \\ \hline
\textbf{Binário} & 0 & 0 & 1 & 1 & 0 & 0 & 0 & 1 &  0 & 0 & 1 & 1 & 0 & 0 & 1 & 0 &  0 & 0 & 1 & 1 & 0 & 0 & 1 & 1 \\ \hline
\textbf{Base 64} & \multicolumn{6}{c|}{12} & \multicolumn{6}{c|}{19} & \multicolumn{6}{c|}{8} & \multicolumn{6}{c|}{51} \\ \hline
\textbf{Texto Base 64} & \multicolumn{6}{c|}{M} & \multicolumn{6}{c|}{T} & \multicolumn{6}{c|}{I} & \multicolumn{6}{c|}{z} \\ \hline
\end{tabular}}
\end{center}
\end{table}
\end{block}
\begin{block}{Exercício b}
Não é seguro, a "segurança" na transformação em base 64 é no sentido de que é seguro transmitir em base 64 porque devido a não existir caracteres de controle não se tem o risco de algum equipamento interpretar erroneamente o conteúdo da mensagem como comandos.
\end{block}
\end{frame}
\begin{frame}
\begin{block}{Exercício c}
Não é possível, embora seja possível aplicar a regra de preencher-se com zero os bits e octetos faltantes a decodificação não sera totalmente confiável.
\end{block}
\end{frame}
\end{document}