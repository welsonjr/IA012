\documentclass[10pt,a4paper]{report}
\usepackage[utf8]{inputenc}
\usepackage[portuguese]{babel}
\usepackage[T1]{fontenc}
\usepackage{amsmath}
\usepackage{amsfonts}
\usepackage{amssymb}
\usepackage{makeidx}
\usepackage{graphicx}
\usepackage{enumerate}
\author{Welson Jr \\ RA: 149802}
\title{Lista de Exercícios II IA012}
\begin{document}
\maketitle
\section*{Exercício 2}
\paragraph{Para saber se uma pessoa tem uma chave secreta K igual à sua, você escolhe uma sequência aleatória de bits A e envia o XOR dela com sua chave. O destinatário recebe a sequência, faz XOR com a chave secreta combinada e devolve o resultado. Se este for igual à sequência aleatória original, então as chaves são iguais. Portanto o teste foi feito sem necessidade de se transmitir nenhuma chave. Mas, existe uma falha no mesmo. Explique qual é.}
\subsection*{R:}
\subparagraph{Embora não haja o envio da chave neste esquema, caso o destinatário venha a conhecer a sequencia original, ele poderá através de simples operações XOR descobrir a chave secreta K do remetente}

\section*{Exercício 4}
\paragraph{ Criptografia e funções hash podem ser usadas para controlar erros na transmissão de uma mensagem. Existe o tipo de controle interno (mensagem e código detetor de erro são concatenados e cifrados) e o externo (mensagem é cifrada antes do código detetor de erros ser calculado e concatenado a ela). Além dos códigos detetores de erros, existem ainda os códigos corretores de erro, os quais não só acusam a ocorrência de algum erro durante a transmissão como também permitem que tais erros sejam corrigidos sem necessidade de uma nova transmissão. Analise a viabilidade de se implantar códigos corretores de erro nos modos interno e externo como descrito acima.}
\subsection*{R:}
\subparagraph{Uma vez que os erros ocorrem durante a transmissão da mensagem, implementar a forma interna impossibilitaria que em caso de corrupção do  bloco cifrado o mesmo possa ser decifrado corretamente e o código de correção de erros seja recuperado e utilizado para recuperar a mensagem.}
\subparagraph{Entretanto caso seja implementada a forma externa, o código de correção de erros pode ser utilizado para recuperar o bloco cifrado antes do processo de decifragem, e com isso os erros de transmissão sejam corrigidos e a mensagem recuperada corretamente}

\section*{Exercício 6}
\paragraph{Há basicamente 3 formas de se usar Nonces:}
\subparagraph{Forma 1:}
\begin{enumerate}[(1)]
\item $A \rightarrow B : N_a$
\item $B \rightarrow A : Ek(N_a)$
\end{enumerate}
\subparagraph{Forma 2:}
\begin{enumerate}[(1)]
\item $A \rightarrow B : Ek(N_a)$
\item $B \rightarrow A : N_a$
\end{enumerate}
\subparagraph{Forma 3:}
\begin{enumerate}[(1)]
\item $A \rightarrow B : Ek(N_a)$
\item $B \rightarrow A : Ek\{f(N_a)\}$
\end{enumerate}
\paragraph{Compare e discuta as 3 formas do ponto de vista da eficácia do Nonce em cumprir sua tarefa.}
\subsection*{R:}
\paragraph{Na forma 1}
\subparagraph{O Nonce é enviado em texto plano e cifrado no retorno com uma chave conhecida entre A e B, como não sofre modificações esta abordagem fornece a um possível atacante pares P C oque possibilitaria a criptoanalise.}
\paragraph{Na forma 2}
\subparagraph{O Nonce é enviado cifrado e retorna decifrado, similarmente a forma 1 esta abordagem possibilita o acumulo de pares P C.}
\paragraph{Na forma 3}
\subparagraph{ Esta forma é a que melhor oferece segurança, o envio encriptado do Nonce impossibilita o atacante de conhecer o conteúdo, e no retorno o Nonce esta modificado impossibilitando o acumulo de pares PC}
\paragraph{Observando esses comportamentos é possível concluir que embora todas as formas cumpram o papel de evitar ataques de repetição, as formas 1 e 2 possibilitam ataques de um criptoanalista que venha a monitorar a comunicação.}


\section*{Exercício 8}
\paragraph{ Em assinaturas digitais intermediadas com criptografia assimétrica temos uma codificação tripla na primeira parte do protocolo. Como o mesmo pode ser modificado de modo a eliminar esta tripla codificação?}
\subsection*{R:}
\subparagraph{A assinatura digital por criptografia assimétrica normalmente utiliza uma mensagem no seguinte formato:}
\begin{enumerate}
\item $X \rightarrow A: ID_X || E_{pr_X}(ID_X||E_{pu_Y}(E_{pr_X}(M))) $
\item $A \rightarrow Y: E_{pr_A}(ID_X||E_{pu_Y}(E_{pr_A}(M))||T)$
\end{enumerate}
\subparagraph{Uma vez que a mensagem esta segura em um bloco que só pode ser aberto por Y, X poderia evitar fazer a codificação do bloco todo ao enviar a requisição para a Autoridade, como a mensagem esta assinada com a chave privada de X, Y terá certeza de quem enviou a mensagem, para acrescentar um pouco mais de segurança X também poderia um carimbo de tempo ao bloco cifrado a fim de evitar um possivel ataque de repetição.}
\begin{enumerate}
\item $X \rightarrow A: ID_X||E_{pu_Y}(E_{pr_X}(M||T)) $
\item $A \rightarrow Y: E_{pr_A}(ID_X||E_{pu_Y}(E_{pr_A}(M))||T)$
\end{enumerate}

\section*{Exercício 10}
\paragraph{ Considere o seguinte esquema de assinatura digital intermediada baseada em chave pública.}
\subparagraph{$Ana \rightarrow Autoridade: ID_A || EK_{prA} [ ID_A || EK_{puB} \{ EK_{prA} ( M ) \} ]$ }
\subparagraph{$Autoridade \rightarrow Beto: EK_{prAut} [ ID_A || EK_{puB} \{ EK_{prA} ( M )\} ]$ }
\paragraph{Explique que problema pode ocorrer caso o remetente tenha mais de um par de chaves e um dos pares tenha sido comprometido.}
\subsection*{R:}
\subparagraph{Supondo que Ana tenha mais de um par de chaves, um deles tenha sido comprometido e não exista um mecanismo ou meio de se notificar a Autoridade certificadora sobre esta falha, um atacante poderia iniciar comunicações com Beto através da Autoridade, assumindo a identidade de Ana pelo canal oficial, e Beto não possuiria meios de descobrir a fraude pois ela seria tratada como legitima do ponto de vista da Autoridade e de Beto}

\section*{Exercício 12}
\paragraph{Considere o seguinte esquema de assinatura digital intermediada baseada em chave pública.} 
\subparagraph{$Ana \rightarrow Autoridade: ID_A || EK_{prA} [ ID_A || EK_{puB} \{ EK_{prA} ( M ) \} ]$}
\subparagraph{$Autoridade \rightarrow Beto:  EK_{prAut} [ ID_A || EK_{puB} \{ EK_{prA} ( M ) \} ]$}
\paragraph{Descreva qual é o tipo de aliança (acordo) entre os envolvidos que poderia surgir a fim de fraudar o sistema. Dica: a fraude se basearia em uma situação tão pouco provável que dificilmente tal aliança teria sucesso.}
\subsection*{R:}
\subparagraph{A autoridade cria um par de chaves (pública e privada) falsas de Beto e divulga a chave pública.
Ana encripta a mensagem com a chave falsa de Beto  e envia para a Autoridade que por sua vez decifra a mensagem com a chave publica de Ana.  Para ocultar a fraude, a Autoridade decifra a mensagem com a chave privada falsa de Beto, cifra-a utilizando a chave pública verdadeira de Beto e envia a mensagem para ele. Assumindo que a Autoridade não tenha falsificado as chaves (pública e privada) de Ana, ela apenas conseguira ler as mensagens mas não conseguira altera-las para enviar a Bob, pois ela não tem a chave privada de Ana.}

\section*{Exercício 14}
\paragraph{Assinatura Digital Intermediada: explique o que é, por que é necessária e dê um exemplo de utilização, mostrando as informações que são trocadas entre as partes.}
\subsection*{R:}
\subparagraph{O Esquema de assinatura Digital Intermediada é similar ao funcionamento de um tabelião. As partes confiam em uma Autoridade que é responsável por certificar a autenticidade da comunicação, evitando assim disputas. Existem basicamente 3 formas e funcionamento:}
\subparagraph{Com criptografia convencional e o arbitro vê a mensagem. }
\begin{enumerate}
\item $X \rightarrow A: M||EK_{XA}([ID_X||H(M)])$
\item $A \rightarrow Y: E_{YA}(ID_X||M||EK_{XA}(ID_X||H(M))) $
\end{enumerate}

\subparagraph{Com criptografia convencional e o arbitro não vê a mensagem. }
\begin{enumerate}
\item $X \rightarrow A: ID_X||E_{XY}(M)||E_{XA}(ID_X||H(E_{XY}(M)||T))$
\item $A \rightarrow Y: E_{YA}(ID_X||E_{XY}(M))||E_{XA}(ID_X||H(E_{XY}(M))||T)$
\end{enumerate}

\subparagraph{Com criptografia de chave publica e o arbitro não vê a mensagem. }
\begin{enumerate}
\item $X \rightarrow A: ID_X||E_{pr_X}(ID_X||E_{pu_Y}(E_{pr_X}(M)))$
\item $A \rightarrow Y: E_{pr_A}(ID_X||E_{pu_Y}(E_{pr_X}(M))||T)$
\end{enumerate}

\section*{Exercício 16}
\paragraph{Existem várias maneiras de se usar funções de hash para autenticar mensagens. Normalmente estas autenticações envolvem o uso de criptografia, seja da mensagem toda ou somente do código hash. Entretanto, o uso de criptografia não é essencial. Mostre como uma função de hash pode ser usada para autenticar e conferir a autenticidade de uma mensagem M sem recorrer aos recursos da criptografia. Considere a função hash como uma caixa preta cujo conteúdo (implementação) é desconhecido.}
\subsection*{R:}
\subparagraph{Uma técnica simples pode ser empregada:}
\begin{enumerate}[(1)]
\item  Primeiro as duas partes combinam entre si uma chave secreta K do mesmo tamanho que o código hash.
\item Então calcula-se o código hash H sobre a mensagem M.
\\ $A: H(M)$
\item Efetua-se uma operação Xor entre a mensagem K e o resultado do calculo do hash e este é o código de autenticação.
\\ $A:MAC = H(M) \oplus K$
\item Concatena-se esse novo código junto com a mensagem M e transmite ao destinatário
\\ $A \rightarrow B: M||MAC$
\item O destinatário recupera o código de autenticação reverte a operação Xor com a chave K, obtendo assim o código hash H
\\ $B:H(M) = MAC \oplus K $
\item Calcula o hash sobre a mensagem e compara com o recebido. Se os dois baterem o destinatário tem certeza de quem enviou a mensagem, pois apenas ele e o emitente possuem a chave K.
\end{enumerate}



\section*{Exercício 18}
\paragraph{O que são chave-mestra e chave de sessão? Quais são as vantagens em utilizá-las?}
\subsection*{R:}
\subparagraph{Chave-mestra é uma chave de longa duração utilizada pelo usurário ou sistema final e o Centro de Distribuição de Chaves.}
\subparagraph{Já a chave de sessão é uma chave obtida temporariamente apenas para uso durante uma conexão logica, sendo descartada posteriormente}
\subparagraph{A vantagem neste esquema esta no fato que a chave mestra é utilizada durante pouquíssimo tempo apenas no inicio da comunicação para se estabelecer uma chave de sessão evitando assim que um possível atacante acumule muitos dados. E o uso posterior de uma chave temporária, a chave de sessão, para a comunicação que é descartada depois do processo, previne que caso um atacante descubra a chave, este possa utilizar a mesma no futuro.}

\section*{Exercício 20}
\paragraph{Explique detalhadamente como funciona o esquema de troca de chaves proposto por Diffie e Helmann em 1976.}
\subsection*{R:}
\subparagraph{Diffie-Hellman propõe o uso de um algoritmo baseado no calculo do logaritmo discreto com base em dois valores públicos $\alpha$ e $p$ para que duas partes negociem uma chave secreta temporária para uma comunicação posterior. Em uma possível negociação entre as partes A e B o algorítimo tem os seguintes passos:}
\begin{enumerate}
\item Existe inicialmente dois números conhecidos publicamente $\alpha$ e $p$. Onde $\alpha < p$ e $\alpha$ é a raiz primitiva no numero primo $p$.
\item Ambas as partes escolhem aleatoriamente um numero X < $p$ para ser suas chaves privadas eo mantem em segredo.
\item Cada parte calcula $Y = \alpha^X \pmod p$ e transmite este numero sua contraparte.
\item A então calcula a chave $K$ com base no valor $Y_B$, $K=(Y_B)^{X_A} \pmod p $
\item B por sua vez faz o mesmo com o valor $Y_A$, $K = (Y_A)^{X_B} \pmod p$
\item com base nas propriedades da aritmética modular A e B chegarão ao mesmo valor para a chave $K$ como pode ser observado a seguir:
\end{enumerate}
\begin{align}
K & = (Y_B)^{X_A} \bmod p \\
& = (\alpha^{X_B} \bmod p)^{X_A} \bmod p \\
& = (\alpha^{X_B})^{X_A} \bmod p  & \textrm{pelas regras da aritmética modular} \\ 
& = \alpha^{X_BX_A} \bmod p \\
& = (\alpha^{X_A} \bmod p)^{X_B} \bmod p \\
& = (Y_A)^{X_B} \bmod p
\end{align}



\section*{Exercício 22}
\paragraph{Sabe-se que é possível criar métodos de criptografia baseados na família de funções matemáticas conhecida como Curvas Elípticas (CE). Mostre os procedimentos para cifrar e decifrar por este algoritmo supondo que fossem usados os pontos da curva $y^2 = x^3 + x + 1 \pmod {23} $ (curva sobre corpo finito primo). Desenvolva um exemplo, mostrando a cifragem e a decifragem da dezena final de seu RA. (Dica: use o algoritmo proposto por Menezes-Vanstone que admite que a mensagem M seja um ponto não pertencente à curva, isto é, ela pode ser formada por um par arbitrário de inteiros.)}
\subsection*{R:}
\subparagraph{Seja:}
\begin{itemize}
\item $y^2 = x^3 + x + 1 \pmod {23}$
\item $Eq(a,b) = Eq_{(23)}(1,1) $
\item $G(13,7)$ 
\end{itemize}
\subparagraph{Para cifrar:}
\begin{itemize}
\item $M_1 = 0$
\item $M_2 = 2$
\item $n_A = 6$
\item $P_A = n_A \times G = 6 \times (13,7) = (13,16)$
\item $k = 4$
\item $kG = 4 \times (13,7) = (17,20)$
\item $kPu = 4 \times (13,7) = (5,19) = (c_1,c_2) $
\item $x_R = M_1 \times c_1 = (0 \times 5) = 0$
\item $y_R = M_1 \times c_2 = (2 \times 19)(\bmod 23) = 15$
\item $c = \{(17,20),(0,15)\}$
\end{itemize}
\subparagraph{Para decifrar:}
\begin{itemize}
\item $n_U = 3$
\item $P_U = n_U \times G = 3 \times (13,7) = (17,3)$
\item $c = \{(17,20),(0,15)\}$
\item $n_u \times kG = 3 \times (17,20) = (5,19) = (c_1,c_2)$
\item $M_1 = x_R \times (c_1)^{-1} = 0 \times (5)^{-1} = 0$
\item $M_2 = x_R \times (c_2)^{-1} = 15 \times (19)^{-1} = 2$
\item $M_1 = 0$
\item $M_2 = 2$
\end{itemize}


\end{document}