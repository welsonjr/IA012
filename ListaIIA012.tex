\documentclass[10pt,a4paper]{report}
\usepackage[utf8]{inputenc}
\usepackage[portuguese]{babel}
\usepackage[T1]{fontenc}
\usepackage{amsmath}
\usepackage{amsfonts}
\usepackage{amssymb}
\usepackage{makeidx}
\usepackage{graphicx}
\usepackage{enumerate}
\author{Welson Jr}
\title{Lista de Exercícios II IA012}
\begin{document}
\maketitle
\section*{Exercício 2}
\paragraph{Para saber se uma pessoa tem uma chave secreta K igual à sua, você escolhe uma sequência aleatória de bits A e envia o XOR dela com sua chave. O destinatário recebe a sequência, faz XOR com a chave secreta combinada e devolve o resultado. Se este for igual à sequência aleatória original, então as chaves são iguais. Portanto o teste foi feito sem necessidade de se transmitir nenhuma chave. Mas, existe uma falha no mesmo. Explique qual é.}
\subsection*{R:}
\subparagraph{Embora não haja o envio da chave neste esquema, caso o destinatário venha a conhecer a sequencia original, ele poderá através de simples operações XOR descobrir a chave secreta K do remetente}

\section*{Exercício 4}
\paragraph{ Criptografia e funções hash podem ser usadas para controlar erros na transmissão de uma mensagem. Existe o tipo de controle interno (mensagem e código detetor de erro são concatenados e cifrados) e o externo (mensagem é cifrada antes do código detetor de erros ser calculado e concatenado a ela). Além dos códigos detetores de erros, existem ainda os códigos corretores de erro, os quais não só acusam a ocorrência de algum erro durante a transmissão como também permitem que tais erros sejam corrigidos sem necessidade de uma nova transmissão. Analise a viabilidade de se implantar códigos corretores de erro nos modos interno e externo como descrito acima.}
\subsection*{R:}
\subparagraph{Uma vez que os erros ocorrem durante a transmissão da mensagem, implementar a forma interna impossibilitaria que em caso de corrupção do  bloco cifrado o mesmo possa ser decifrado corretamente e o código de correção de erros seja recuperado e utilizado para recuperar a mensagem.}
\subparagraph{Entretanto caso seja implementada a forma externa, o código de correção de erros pode ser utilizado para recuperar o bloco cifrado antes do processo de decifragem, e com isso os erros de transmissão sejam corrigidos e a mensagem recuperada corretamente}

\section*{Exercício 6}
\paragraph{Há basicamente 3 formas de se usar Nonces:}
\subparagraph{Forma 1:}
\begin{enumerate}[(1)]
\item $A \rightarrow B : N_a$
\item $B \rightarrow A : Ek(N_a)$
\end{enumerate}
\subparagraph{Forma 2:}
\begin{enumerate}[(1)]
\item $A \rightarrow B : Ek(N_a)$
\item $B \rightarrow A : N_a$
\end{enumerate}
\subparagraph{Forma 3:}
\begin{enumerate}[(1)]
\item $A \rightarrow B : Ek(N_a)$
\item $B \rightarrow A : Ek\{f(N_a)\}$
\end{enumerate}
\paragraph{Compare e discuta as 3 formas do ponto de vista da eficácia do Nonce em cumprir sua tarefa.}
\subsection*{R:}
\paragraph{Na forma 1}
\subparagraph{O Nonce é enviado em texto plano e cifrado no retorno com uma chave conhecida entre A e B, como não sofre modificações esta abordagem fornece a um possível atacante pares P C oque possibilitaria a criptoanalise.}
\paragraph{Na forma 2}
\subparagraph{O Nonce é enviado cifrado e retorna decifrado, similarmente a forma 1 esta abordagem possibilita o acumulo de pares P C.}
\paragraph{Na forma 3}
\subparagraph{ Esta forma é a que melhor oferece segurança, o envio encriptado do Nonce impossibilita o atacante de conhecer o conteúdo, e no retorno o Nonce esta modificado impossibilitando o acumulo de pares PC}
\paragraph{Observando esses comportamentos é possível concluir que embora todas as formas cumpram o papel de evitar ataques de repetição, as formas 1 e 2 possibilitam ataques de um criptoanalista que venha a monitorar a comunicação.}


\section*{Exercício 8}
\paragraph{ Em assinaturas digitais intermediadas com criptografia assimétrica temos uma codificação tripla na primeira parte do protocolo. Como o mesmo pode ser modificado de modo a eliminar esta tripla codificação?}

\section*{Exercício 10}
\paragraph{ Considere o seguinte esquema de assinatura digital intermediada baseada em chave pública.}
\subparagraph{$Ana \rightarrow Autoridade: ID_A || EK_{prA} [ ID_A || EK_{puB} \{ EK_{prA} ( M ) \} ]$ }
\subparagraph{$Autoridade \rightarrow Beto: EK_{prAut} [ ID_A || EK_{puB} \{ EK_{prA} ( M )\} ]$ }
\paragraph{Explique que problema pode ocorrer caso o remetente tenha mais de um par de chaves e um dos pares tenha sido comprometido.}

\section*{Exercício 12}
\paragraph{Considere o seguinte esquema de assinatura digital intermediada baseada em chave pública.} 
\subparagraph{$Ana \rightarrow Autoridade: ID_A || EK_{prA} [ ID_A || EK_{puB} \{ EK_{prA} ( M ) \} ]$ }
\subparagraph{$Autoridade \rightarrow Beto:  EK_{prAut} [ ID_A || EK_{puB} \{ EK_{prA} ( M ) \} ]$}
\paragraph{Descreva qual é o tipo de aliança (acordo) entre os envolvidos que poderia surgir a fim de fraudar o sistema. Dica: a fraude se basearia em uma situação tão pouco provável que dificilmente tal aliança teria sucesso.}

\section*{Exercício 14}
\paragraph{Assinatura Digital Intermediada: explique o que é, por que é necessária e dê um exemplo de utilização, mostrando as informações que são trocadas entre as partes.}
\paragraph{R:}

\section*{Exercício 16}
\paragraph{Existem várias maneiras de se usar funções de hash para autenticar mensagens. Normalmente estas autenticações envolvem o uso de criptografia, seja da mensagem toda ou somente do código hash. Entretanto, o uso de criptografia não é essencial. Mostre como uma função de hash pode ser usada para autenticar e conferir a autenticidade de uma mensagem M sem recorrer aos recursos da criptografia. Considere a função hash como uma caixa preta cujo conteúdo (implementação) é desconhecido.}

\section*{Exercício 18}
\paragraph{O que são chave-mestra e chave de sessão? Quais são as vantagens em utilizá-las?}

\section*{Exercício 20}
\paragraph{Explique detalhadamente como funciona o esquema de troca de chaves proposto por Diffie e Helmann em 1976.}
\section*{R:}
\subparagraph{Diffie-Hellman propõe o uso de um algoritmo baseado no calculo do logaritmo discreto com base em dois valores públicos $\alpha$ e $p$ para que duas partes negociem uma chave secreta temporária para uma comunicação posterior. Em uma possível negociação entre as partes A e B o algorítimo tem os seguintes passos:}
\begin{enumerate}
\item Existe inicialmente dois números conhecidos publicamente $\alpha$ e $p$. Onde $\alpha < p$ e $\alpha$ é a raiz primitiva no numero primo $p$.
\item Ambas as partes escolhem aleatoriamente um numero X < $p$ para ser suas chaves privadas eo mantem em segredo.
\item Cada parte calcula $Y = \alpha^X \pmod p$ e transmite este numero sua contraparte.
\item A então calcula a chave $K$ com base no valor $Y_B$, $K=(Y_B)^{X_A} \pmod p $
\item B por sua vez faz o mesmo com o valor $Y_A$, $K = (Y_A)^{X_B} \pmod p$
\item com base nas propriedades da aritmética modular A e B chegarão ao mesmo valor para a chave $K$ como pode ser observado a seguir:
\end{enumerate}
\begin{align}
K & = (Y_B)^{X_A} \bmod p \\
& = (\alpha^{X_B} \bmod p)^{X_A} \bmod p \\
& = (\alpha^{X_B})^{X_A} \bmod p  & \textrm{pelas regras da aritmética modular} \\ 
& = \alpha^{X_BX_A} \bmod p \\
& = (\alpha^{X_A} \bmod p)^{X_B} \bmod p \\
& = (Y_A)^{X_B} \bmod p
\end{align}



\section*{Exercício 22}
\paragraph{Sabe-se que é possível criar métodos de criptografia baseados na família de funções matemáticas conhecida como Curvas Elípticas (CE). Mostre os procedimentos para cifrar e decifrar por este algoritmo supondo que fossem usados os pontos da curva y2 = x3 + x + 1 mod 23 (curva sobre corpo finito primo). Desenvolva um exemplo, mostrando a cifragem e a decifragem da dezena final de seu RA. (Dica: use o algoritmo proposto por Menezes-Vanstone que admite que a mensagem M seja um ponto não pertencente à curva, isto é, ela pode ser formada por um par arbitrário de inteiros.)}
\end{document}